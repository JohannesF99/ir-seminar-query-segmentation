\section{Evaluation} \label{evaluation}

The paper by \citeauthor{Risvik:2003} works as a good starting point for query segmentation. It covers the basic concepts to understand most segmentation approaches. Additionally, the authors introduce an easy to understand approach with their connexity score computation. The paper does not include any results or evaluation metrics itself, which could be helpfull for potential readers. \citeauthor{Risvik:2003} only mention the query logs they used for their experiments in the conclusion of the original paper, which could be included earlier. They also focus heavily on the efficiency of their experiment setup, where most paper lay the focus on the effectiveness of an approach. In general, the approach works well with most common words and segments but probably falls off for new or rare but important words that most of the time have no high frequency or mutual information. 
An interesting part of the paper was the discussion about problems for long time usage of this approach, more specifically the inability to update entries selectively\footnote{e.g. for the score recomputation when adding new query logs to the corpus}. Adding new entries to the query logs would result in a full rebuild of the database for every update. \citeauthor{Risvik:2003} claimed that a even though this solution is not perfect, because the process takes roughly three hours, the task is in a manageable time frame and can be run on a daily basis.


The second approach by \citeauthor{Bergsma:2007} reiterates and expands on some of the concepts like mutual information covered by \citeauthor{Risvik:2003}. This is then used to introduce the reader to their SVM-based approach. There is no short introduction to supervised machine learning or SVMs in general, but they always reference the original paper, so the reader can research the missing concepts on their own. \citeauthor{Bergsma:2007} include examples for most of their features, which makes it easy to understand. As seen in Table \ref{table-segmentation-performance-bergsma-2007}, their approach significantly improves current state-of-the-art baselines like \citeauthor{Risvik:2003}'s approach. Interestingly, the dependecy features don't improve the effectiveness that much and in some cases even decrease it.
\citeauthor{Bergsma:2007}'s paper also provides further information for the connexity score approach. They describe how the computation for mutual information works and present some results in Table \ref{table-segmentation-performance-bergsma-2007}, which the original paper lacks. So the chosen paper build upon each other and provide an overview over the the topic query segmentation in general as well as some possible approaches.
