\section{Introduction} \label{introduction}
When a query, defined as a sequence of tokens, is put into a search engine, the process of query interpretation and document retrieval starts, with the goal of finding documents relevant to the information need of the user. The query can also contain words that do not contribute to improving the understanding of the information need, because user input is often expressed in natural language. As common with natural language, words do not stand alone but are connected through syntactic and semantic relationships. The same applies for queries. Modern search engines take this characteristic into account. For any given query the relevant documents and the order in that they are represented will change depending on the order of the input query tokens. The user therefore can make active use of query segmentation while searching the web. The most popular way to prioritize a given segmentation on a search engine is the usage of quotation marks. This way, instead of letting the retrieval system guess which segmentation is correct, the user can provide the ground truth within the query. In reality, the majority of the users do not use this feature, making automatic query segmentation an important part of query understanding in general. 

Applications of segmentation include improved retrieval accuracy of relevant documents and filtering out non-relevant results for the actual query. Additionally, there are advanced techniques like query substitution and query expansion that can help improve retrieval effectiveness. These techniques are very important for query understanding in general but not in scope for this paper.

First, an approach by \citet{Risvik:2003} is presented, followed by another approach by \citet{Bergsma:2007}, with the goal of demonstrating how automatic query segmentation can be realized in retrieval systems.