\title{Query segmentation using mutual information scores and supervised classification models}

\author{Johannes Franke}
\affiliation{%
    \institution{Friedrich-Schiller-Universit�t Jena}
    \city{Jena}
    \country{Germany}
}
\email{johannes.franke@uni-jena.de}

\begin{abstract}
My paper sets the focus on query segmentation as one tool to address and improve query understanding and analysis using a score computation approach by \citet{Risvik:2003} as well as a SVM-based segmentation approach by \citet{Bergsma:2007}. 

In order to give a well-rounded summary of this topic, two prominent approaches for improving query segmentation were selected. The first approach by \citet{Risvik:2003} was published in 2003 and introduces a measurement called \textit{connexity}, while the second approach was published in 2007 by \citet{Bergsma:2007} and treats segmentation as a classification task using support vector machines (SVMs). After an explanation on how the approaches work, the advantages and disadvantages of the approaches are discussed.
\end{abstract}

\keywords{Seminar IR, Winter term 2024/25, query segmentation, connexity, SVM}

\maketitle
