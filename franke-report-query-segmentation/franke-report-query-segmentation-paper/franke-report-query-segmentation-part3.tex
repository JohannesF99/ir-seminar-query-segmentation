\section{segmentation classification using SVMs} \label{approach2}

\begin{table}[t]
 \caption{Indicator features\\for all words in the decision boundary}
 \label{table-indicator-features-bergsma-2007}
 \centering
 \small
 \begin{tabular}{@{}ll@{}}
  \toprule
  Feature name & Feature description \\
  \midrule
  is-the & token $x$ is "the"\\
  is-free & token $x$ is "free"\\
  POS-tags & Part-of-speech tags of pair $(x_{L0},x_{R0})$\\
  fwd-pos & position $i$ from beginning\\
  rev-pos & position $N-i$ from end\\
  \bottomrule
 \end{tabular}
\end{table}

\begin{table}[t]
 \caption{Statistical features for words in the decision boundary}
 \label{table-statistical-features-bergsma-2007}
 \centering
 \small
 \begin{tabular}{@{}ll@{}}
  \toprule
  Feature & description\\
  \midrule
  web count & web frequency of 'x'\\
  pair count & web frequency of 'w x'\\
  definite & web frequency of 'the w x'\\
  collapsed & web frequency of 'wx' (one word)\\
  and count & web frequency of 'w and x'\\
  genitive & web frequency of 'w`s x'\\
  Qcount-1 & frequency of 'x' in query logs\\
  Qcount-2 & frequency of 'w x' in query logs\\
  \bottomrule
 \end{tabular}
\end{table}

\begin{table*}[t]
 \caption{Segmentation accuracy comparison of state-of-the-art baselines and the approach by \citet{Bergsma:2007}}
 \label{table-segmentation-performance-bergsma-2007}
 \centering
 \small
 \begin{tabular}{@{}llcccc@{}}
  \toprule
  \multirow{2}{*}{Feature type} & \multirow{2}{*}{Feature span} & \multicolumn{2}{c}{Test set} & \multicolumn{2}{c}{Intersection set} \\
  && Seg-Acc & Qry-Acc & Seg-Acc & Qry-Acc\\
  \midrule
  MI & Decision-Boundary & 68.0 & 26.6 & 73.8 & 34.7\\
  \midrule
  Basic & Decision-Boundary & 71.7 & 29.2 & 77.6 & 39.7\\
  Basic & Decision-Boundary, Context & 80.2 & 52.0 & 85.6 & 62.1\\
  Basic & Decision-Boundary, Context, Dependency & 81.1 & 53.2 & 86.2 & 64.8\\
  \midrule
  All & Decision-Boundary & 84.3 & 57.8 & 86.6 & 63.5\\
  All & Decision-Boundary, Context & 86.3 & 63.8 & 89.2 & 71.7\\
  All & Decision-Boundary, Context, Dependency & 85.8 & 61.0 & 88.7 & 69.4\\
  \bottomrule
 \end{tabular}
\end{table*}

\paragraph*{Introduction}
After the discussion on the early segmentation approach using connexity scores defined by \citeauthor{Risvik:2003}, this chapter is used to introduce yet another approach for query segmentation by \citeauthor{Bergsma:2007}. Similar to \citeauthor{Risvik:2003}, \citeauthor{Bergsma:2007} start with the definition of query interpretation in general. When a query, defined as a sequence of tokens, is put into a search engine, it undergoes the known process of interpretation and document retrieval, often resulting in documents relevant to the topics but containing additional unrelated words. The user input is often expressed in natural language. Because of that, the query tokens do not stand alone but are connected through syntactic and semantic relationships. Modern search engines take this characteristic also into account. For any given query the relevant documents and the order in that they are represented will change depending on the order of the input query tokens. The user therefore can make active use of query segmentation while searching the web. The most popular method is the usage of quotation marks to force search engines to prioritize a given segmentation. This way, instead of letting the retrieval system guess which segmentation is correct, the user can provide the ground truth within the query. In reality, the majority of the users do not use this feature, making query segmentation an important part of query understanding in general. Applications of segmentation include faster retrieval of relevant documents and filtering out irrelevant results for the actual query. Additionally, there are advanced techniques like query substitution and query expansion that can help improve retrieval effectiveness - techniques that are very important for query understanding but not in scope for this paper. The process of query segmentation by \citeauthor{Bergsma:2007} is implemented using a data-driven, supervised machine learning model, where the segmentation itself is treated as a classification task for each potential split point in the token sequence. This is a new and modern approach to the classic feaute-based approaches like the connexity score computation by \citet{Risvik:2003}.

\paragraph*{Related Work}
The discussed approach by \citet{Risvik:2003}, the research on noun compound (NC) bracketing by \citet{Nakov:2005}, and the feature analysis of \citet{Keller:2003} were some of the main contributions to \citeauthor{Bergsma:2007}'s segmentation system. \citeauthor{Keller:2003} were one of the first to introduce web-derived features as an alternative to corpus-based features to improve effectiveness. \citeauthor{Nakov:2005} on the other hand suggested to look "beyond the n-gram" for co-occurrence counts (like "$w$'s $x$" or "$wx$" in text for words $w$ and $x$).
\citeauthor{Nicholson:2005} additionally added lexical pattern features involving determiners (like "the $w$ $x$" for words $w$ and $x$). Aside from these syntacitc relationships, modern segmentation models also integrate semantics in their NC interpretation process, improving effectiveness even further \cite{Girju:2005}.

\paragraph*{Methodology}
\subparagraph*{\textbf{Segmentation as Classification Task}}
Similiar to the approach in chapter 2, \citeauthor{Bergsma:2007} define a segmentation as a mapping $S: x \to y$, where $x$ is a query of $N$ tokens with segmentation $y$. The segmentation is selected from $Y_N$, the set of all possible segmentations with cardinality $|Y_N| = 2^{N-1}$. They then use supervised machine learning to train a model to guess the optimal segmentation $\hat{y}$. The input for the model is a list $T = \{(x_0, y_0), ..., (x_i, y_i)\}$ where $x_i$ is a query and $y_i$ their best possible segmentation. For every pair exist a set of features $\Psi(x,y)$ that characterizes the segmentation for the corresponding query. The machine learning model then tries to maximize the score:
\begin{align*}
	\hat{y} &= argmax_y(Score_w(x,y)),
\end{align*}
where the score for a weighting $w$ is defined as $Score_w(x,y) = w \cdot \Psi(x,y)$. \citeauthor{Bergsma:2007} implement this using Support Vector Machines (SVMs), that were first described by \citet{Joachims:2002}. While alternatives like Hidden Markov\footnote{\todo{short description Hidden Markov model}} models were tested, traditional SVMs had better effectiveness in comparison. The segmentation pipeline then evaluates the $N-1$ potential breakpoints for every query of length $N$, based on the token feature-set.

\subparagraph*{\textbf{Features}}
As before mentioned, the SVM is trained for every query $x$ and corresponding ground truth $y$ with a set of features $\Psi(x,y)$, where every breakpoint is a seperate classification task. To describe a classification task where the potential breakpoint lays between the words $x_{L0}$ and $X_{R0}$, they use the following terminology:
\begin{align*}
\{..., X_{L2}, X_{L1}, X_{L0}, X_{R0}, X_{R1}, X_{R2},...\},
\end{align*}
with $L,R$ describing the position and distance of the word relative to the breakpoint.
The naive implementation for this task would only consider the features for the words left and right from the breakpoint. \citeauthor{Bergsma:2007} improved this version to include features for all words within a given window. This window is called \textit{decision boundary}. The size of these boundaries was set empiricistically to three.
They also differentiate between \textit{decision bound-} and \textit{context-}features. While decision bound features only include features for the corresponding left and right tokens of a breakpoint, context features take the surrounding tokens of the decision boundary into account.

Decision bound features are again split up into 2 groups, indicator features and statistical features. As seen in \ref{table-indicator-features-bergsma-2007}, the authors used Part-of-Speech (POS) tagging as well as the position of the token as indicator features. Some unusual features are the \textit{is-free} and \textit{is-the} features. Those are included because these words in many cases introduce or finish segments and therefore inpove accuracy.
Statistical features are shown in \ref{table-statistical-features-bergsma-2007} and use mutual information (MI) similar to \citeauthor{Risvik:2003}. They additionally add some background to the computation of the MI score: For $x_{L0},x_{R0}$, the $MI$-value computes with
\begin{align*}
MI &= \log \frac{Pr(x_{L0}x_{R0})}{Pr(x_{L0}) \cdot Pr(x_{R0})} \\
&\Leftrightarrow \log C(x_{L0}x_{R0}) + \log K - \log C(x_{L0}) - \log C(x_{R0}).
\end{align*}
As seen in the equation, instead of the usual proximity metric, they adapted to metrics that are common in information retrieval. The mutual information for two token $x_{L0},x_{R0}$ depends on the number of all documents $K$ in the corpora and the number of documents where certain token appear $C$.
\todo{short explaination mutual information}
Other statistical features include web-counts of single token or token pairs.

\todo{context features}\\
\todo{dependency features}

\paragraph*{Experimental Setup}
We used the AOL Search Query Log (Pass et al., 2006), focusing on queries associated with user clicks for reliable relevance signals. After filtering, the dataset included 1,500 queries (500 for each of training, validation, and testing). Ground truth was annotated manually, yielding moderate agreement ($\kappa=0.69$) and highlighting the challenge of query ambiguity.

4.2 Evaluation
Metrics included segmentation accuracy (Seg-Acc) and query-level accuracy (Qry-Acc). We employed McNemar's test with $p<0.05$ to assess significance.

\paragraph*{Results}
As seen in Table \ref{table-segmentation-performance-bergsma-2007}, the model outperformed baselines:
\begin{enumerate}
\item[1)] Naive Baseline 1 (always split): 44.8\% Seg-Acc, 4.2\% Qry-Acc.
\item[2)] Naive Baseline 2 (never split): 55.2\% Seg-Acc, 4.0\% Qry-Acc.
\item[3)] State-of-the-Art Baseline (MI): 68\% Seg-Acc, 26.6\% Qry-Acc.
\end{enumerate}
By optimizing feature weights through SVMs, our method achieved significant improvements, despite the moderate inter-annotator agreement in ground truth.

\paragraph*{Conclusion}
This study demonstrated the efficacy of a data-driven supervised approach to query segmentation. By leveraging discriminative features and context-aware models, the proposed method surpasses traditional MI-based baselines, enhancing search relevance and efficiency. Future work will explore deeper semantic representations and scaling for larger datasets.